\documentclass{beamer}
  
  \usepackage[utf8]{inputenc}
  \usetheme{Copenhagen}
  
  \title{Software Tools for Reproducible Research}
  \author{Alfonso Landeros}
  \date{February 6, 2020}
  
\begin{document}

% frame each software tool in terms of "a job to be done"
\frame{\titlepage}

\begin{frame}{Overview}
  \begin{itemize}
    \item Why does this matter?

    \item Version Control

    \item Interactive Computing Environments

    \item Free Open-Source Software

  \end{itemize}
\end{frame}

%%%%% Why does this matter? %%%%%

%%%%% on a practical level... %%%%%

\begin{frame}{Why does this matter?}
  Researchers often need to manage a number of documents: 
  \begin{itemize}
    \item manuscripts

    \item notes

    \item datasets

    \item images

    \item scripts
  \end{itemize}

  Today, many of those documents live on a computer.
\end{frame}

\begin{frame}{Why does this matter?}
  \center
  \includegraphics[width=0.8\textheight]{duke-potti-scandal}

  \begin{itemize}
    \item \tiny Potti et al (2006) Genomic signatures to guide the use of chemotherapeutics, Nature Medicine, 12(11):1294--1300.
  \end{itemize}
  

  Good work habits help one achieve quality research.
\end{frame}

\begin{frame}{From Baggerly and Coombes (2009)}
  \begin{quotation}
    \noindent High-throughput biological assays such as microarrays let us ask very detailed questions about how diseases operate, and promise to let us personalize therapy.
    \textbf{Data processing, however, is often not described well enough to allow for exact reproduction of the results, leading to exercises in “forensic bioinformatics”} where aspects of raw data and reported results are used to infer what methods \textbf{must} have been employed.
  \end{quotation}
\end{frame}

\begin{frame}{Why does this matter?}
  \center
  \includegraphics[width=0.8\textwidth]{git-log-example}

  Documenting one's work enhances collaboration.
\end{frame}

%%%%% on a philosophical level... %%%%%

\begin{frame}{Why does this matter?}

  \begin{quote}
  An article about computational science in a scientific publication is \textbf{not} the scholarship itself, it is merely \textbf{advertising} of the scholarship.
  The actual scholarship is the complete software development environment and the complete set of instructions which generated the figures.
  \end{quote}
  \center -- Buckheit and Donoho (1995)
  \hfill \\

  \center
  Reproducibility is fundamental to science.
\end{frame}

%%%%% Version Control %%%%%

\begin{frame}{Version Control}
  \center
  \includegraphics[height=0.8\textheight]{why-version-control}
\end{frame}

\begin{frame}{Version Control}
  \begin{itemize}
    \item Tracking different \textit{versions} of a document by hand is tedious and error-prone.

    \item Tracking the \textit{history} of a changing document is cumbersome.
  \end{itemize}
\end{frame}

\begin{frame}{Git -- Distributed Version Control}
  \center \includegraphics[width=0.5\textwidth]{git-example}
  \begin{itemize}
    \item Originally developed by Linus Torvaldis for managing the Linux kernel.

    \item \textit{Distributed} in the sense that multiple people own and can edit copies of a repository.

    \item Git operates on \textit{changes} between versions of a document.

    \item Each version of a respository exists as a \textit{commit}, or snapshot.
  \end{itemize}
\end{frame}

\begin{frame}{Git -- Distributed Version Control}
  \center
  \includegraphics[height=0.8\textheight]{git-model}
\end{frame}

\begin{frame}{Git -- Distributed Version Control}
  \begin{columns}
    \begin{column}{0.5\textwidth}
      \begin{itemize}
        \item C: add a remote repository for backup
        \item D: push the entire history to the remote
        \item E: make local changes and commit them
        \item F: push the changes to the remote
      \end{itemize}
    \end{column}

    \begin{column}{0.5\textwidth}
      \begin{figure}
        \center
        \includegraphics[height=0.75\textheight]{git-example-actions}
      \end{figure}
    \end{column}
  \end{columns}
\end{frame}

\begin{frame}{Using Git Effectively}
  \begin{figure}
    \center
    \includegraphics[width=0.7\textheight]{using-git-effectively}
  \end{figure}

  \begin{itemize}
    \item Commit \textit{early} and \textit{often}.

    \item Write informative commit messages.

    \item Be judicious about what you commit: ``raw'' vs ``binary'' files
  \end{itemize}
\end{frame}

\begin{frame}{What should I Version Control?}
  Generally, ``raw'' files are better suited for Git because it can track changes to individual bytes.

  \hfill \\

  \begin{columns}
    \begin{column}{0.5\textwidth}
      \center{Good}
      \begin{itemize}
        \item source files

        \item \LaTeX\ files

        \item Markdown
      \end{itemize}
    \end{column}

    \begin{column}{0.5\textwidth}
      \center{``Bad''}
      \begin{itemize}
        \item MS Office files

        \item binary executables

        \item images
      \end{itemize}
    \end{column}
  \end{columns}

  \hfill \\
  
  \textbf{Note for Manuscripts}: Place sentences on \textit{separate} lines so that Git can track changes.
\end{frame}

\begin{frame}{Diff Example}
  \center
  \includegraphics[width=0.7\textwidth]{diff-example}
\end{frame}

% working directory - the folder on your computer
%
% staging area - files that you tag to commit changes
%   are said to be staged / in the staging area
% 
% Each commit has an associated message - this is crucial!
%   a commit message should tell you what was changed and why
%
% local repo - your working repository; the collection of all the files in your folder + history + changes
%
% remote repo - a foreign version that lives on a server somewhere; usually your backup
%
% git add - tag for changes
% git commit - apply the changes to the repo
% git checkout - "checkout" some version of the repo
% git merge - "combine" changes from different branches
% git push - send the changes to your backup
% git fetch - check for any changes in remote, but do not apply them to your repo
\begin{frame}{Web-based Version Control Services}
  \begin{figure}
    \center
    \includegraphics[width=0.2\textwidth]{github_mark}
    \includegraphics[width=0.2\textwidth]{gitlab_mark}
  \end{figure}

  \begin{itemize}
    \item Provides a central hub for a Git repository.

    \item Great way to share your work with others, or backup your local working directory.

    \item Provides an interface for discussing a project: issue tracking and documentation.

    \item Both private and public options available.
  \end{itemize}
\end{frame}

\begin{frame}{GitHub Example}
  \center
  \includegraphics[width=0.9\textwidth]{github-example}
\end{frame}

\begin{frame}{}
  \center
  \includegraphics[height=0.9\textheight]{frustrating-git}
\end{frame}

\begin{frame}{GitKraken -- Git without the Command Line}
  \center \includegraphics[width=0.3\textwidth]{gitkraken_mark}

  \begin{itemize}
    \item Provides a button for the most used Git commands.

    \item Cleaner visualizations compared to the Git defaults.

    \item However, it is not as flexible as the command line tool.
  \end{itemize}
\end{frame}

\begin{frame}{GitKraken -- Git without the Command Line}
  \center
  \includegraphics[width=\textheight]{gitkraken-demo}
\end{frame}

\begin{frame}{Pachyderm}
  \center
  \includegraphics[width=0.5\textwidth]{pachyderm-graph}
  \begin{itemize}
    \item Version control for data and data analysis.

    \item The premise is that datasets may change over time and feed into complex analysis pipelines.

    \item Pachyderm provides both services and the infrastructure to support data versioning.
  \end{itemize}
\end{frame}

\begin{frame}{Pachyderm - Version Control}
  \center
  \includegraphics[width=0.5\textwidth]{pachyderm-version}
  \begin{itemize}
    \item Ensures that all collaborators are working with the same datasets.
  \end{itemize}
\end{frame}

\begin{frame}{Pachyderm - Pipelines}
  \center
  \includegraphics[width=0.5\textwidth]{pachyderm-infrastructure}
  \begin{itemize}
    \item Provides computing environment equipped with all the resources needed to reproduce outputs.

    \item Parallelism is built-in to the service.
  \end{itemize}
\end{frame}

\begin{frame}{Pachyderm - Data Provenance}
  \center
  \includegraphics[width=0.4\textwidth]{pachyderm-provenance}
  \begin{itemize}
    \item Allows one to identify the changes to datasets and processing pipelines that produce different results.
  \end{itemize}
\end{frame}

% Git - version control for "raw" files
% -- automates versioning of documents

% GitKraken - GUI for Git
% -- Git without the command line
% -- visual representation of your repository
% -- remove burden of remembering every command

% GitHub - online version of Git
% -- provides a backup of your work history

% GitLab - same as GitHub
% -- also provides a private version that can be used by an ambitious team

% Pachyderm - version control for data analysis
% -- keep data analysis workflows sane and easy to debug

% Applications - writing a paper!
% -- what should be versioned?
% -- how can you share it with collaborators?
% -- how can you track changes in your documents?

\begin{frame}{Interactive Computing Environments}
  There is a sizeable amount of diversity in the machines used for scientific work:
  \begin{itemize}
    \item operating systems

    \item hardware components

    \item software resources

    \item software versions
  \end{itemize}
\end{frame}

% Jupyter / Jupyter Notebook/Lab - living, self-contained documents
% -- 

\begin{frame}{Jupyter -- living, self-contained documents}
  \center
  \includegraphics[width=0.7\textwidth]{jupyter-example}
\end{frame}

\begin{frame}{Jupyter -- living, self-contained documents}
  \begin{itemize}
    \item Project that grew out of the IPython environment created by Fernando Perez.

    \item Supports multiple programming and scripting languages.

    \item Jupyter \textbf{notebooks} are interactive documents that execute a long script.

    \item Each document is made up of \textit{cells}: individual blocks of code or text.

    \item The document is annotated using Markdown and \LaTeX.
  \end{itemize}
\end{frame}

\begin{frame}{Markdown Cell Example}
  \center
  \includegraphics[width=\textwidth]{markdown-example2}
\end{frame}

\begin{frame}{Markdown Cell Example}
  \center
  \includegraphics[width=\textwidth]{markdown-example1}
\end{frame}

\begin{frame}{JupyterLab}
  \center
  \includegraphics[width=0.8\textwidth]{jupyterlab-example}
  \begin{itemize}
    \item An integrated environment that allows one to manage multiple notebooks, source files, and processes.
  \end{itemize}
\end{frame}

\begin{frame}{Putting Everything Together}
  \begin{itemize}
    \item Research Journal

    \item Manuscripts

    \item Thesis Project

    \item Personal Website
  \end{itemize}
\end{frame}

\begin{frame}{Example: Personal Website}
  \center
  \includegraphics[width=\textwidth]{personal-page-example1}
\end{frame}

\begin{frame}{Example: Personal Website}
  \center
  \includegraphics[width=\textwidth]{personal-page-example2}
\end{frame}

\begin{frame}{Free Open-Source Software vs Proprietary Software}
  \begin{itemize}
    \item Many of the products listed here are not only free, but \textit{open-source}.

    \item Anyone with a computer and internet access can view, modify, and build the source code.

    \item Commercial software rarely has this level of transparency.
  \end{itemize}
\end{frame}

% Motivating Examples: Numerical Recipes, MATLAB, Mathematica

% Interdependence; example: MacOS

% Main Takeaways:
%
% 1. Researchers often need to manage a number of documents - notes, manuscripts, analysis / computational code.
%
% 2. Software engineers have accrued a number of tools that automate many aspects of their workflow.
% Some of these tools have overlap with a scientist's workflow, especially in an age where scientific computing and data analysis are front-and-center.
%
% 3. ???
%

\begin{frame}{Acknowledgements}
  For forcing me to learn Git:
  \begin{itemize}
    \item Dr.\ Marc Suchard
    \item Dr.\ Hua Zhou
  \end{itemize}

  For actually learning Git and practicing reproducibility:
  \begin{itemize}
    \item BIOMATH 204: Biomedical Data Analysis
    \item BIOSTAT M280: Statistical Computing
  \end{itemize}

  \center Thank you!
\end{frame}

\end{document}
