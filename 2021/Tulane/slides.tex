% !TEX program = xelatex
\documentclass[8pt]{beamer}
\useoutertheme{metropolis}
\useinnertheme{metropolis}
\usefonttheme{metropolis}
\usecolortheme{owl}

\usepackage{animate}
\usepackage{blkarray}
\usepackage{booktabs}
\usepackage{graphicx}
\usepackage{mathtools}

% theme colors
\setbeamercolor{background canvas}{bg=black!80}
\setbeamercolor{frametitle}{bg=black!70}

% \addbibresource{references.bib}

% %remove the icon
% \setbeamertemplate{bibliography item}{}

% %remove line breaks
% \setbeamertemplate{bibliography entry title}{}
% \setbeamertemplate{bibliography entry location}{}
% \setbeamertemplate{bibliography entry note}{}

% hyperref options
\hypersetup{
    colorlinks=true,
    linkcolor=blue,
    urlcolor=red,
}

% phantom characters for tidy typesetting
\newcommand{\phm}{\hphantom{+}}

% setting derivatives
\newcommand{\ddt}[1]{\frac{\mathrm{d}#1}{\mathrm{d}t}}

% titlepage
\title{An Examination of Reopening Strategies}
\author{Alfonso Landeros, Department of Computational Medicine @ UCLA}
\date{February 26, 2021}

% fix overfull \vbox in metropolis theme
\def\titlepage{%
  \usebeamertemplate{title page}%<---
}

\begin{document}
\maketitle

%%%%% presentation outline %%%%%
\begin{frame}{Outline}
\begin{itemize}
    \item Background
    \item (Simplifying) modelling assumptions
    \item Stratefied SEIR model
    \item Robustness of transmission assumptions
    \item Impact of mitigation strategies
    \item Simulating school reopening
\end{itemize}
\end{frame}

%%%%% Background %%%%%
\section{Background}

\begin{frame}{Background: Demographic and epidemiologic characteristics}
%
\begin{itemize}
    \item School districts have median 22\% of local population under 18 years of age~\footnotemark[1].
    \item Cumulative cases in children: August 6, 2020 9.1\% / February 4, 2021 12.9\%~\footnotemark[2].
    \item Rare complication: Multisystem Inflammatory Syndrome in Children (MIS-C)~\footnotemark[3].
    \item Evidence that children have lower susceptibility compared to adults~\footnotemark[4].
    \begin{itemize}
        \item Adolescents may be more similar to adults.
    \end{itemize}
    \item Children typically present with milder symptoms~\footnotemark[5], which makes detection and surveillance challenging.
\end{itemize}

\footnotetext[1]{National Center for Education Statistics. ACS-ED 2014-2018.}
\footnotetext[2]{Academy of American Pediatricians \& Children's Hospital Association. State data report. https://services.aap.org/en/pages/2019-novel-coronavirus-covid-19-infections/children-and-covid-19-state-level-data-report/}
\footnotetext[3]{Health Alert Network. MIS-C Associated with COVID-19. https://emergency.cdc.gov/han/2020/han00432.asp}
\footnotetext[4]{Viner RM et al. Susceptibility to SARS-CoV-2 Infection Among Children and Adolescents Compared With Adults: A Systematic Review and Meta-analysis. JAMA Pediatr. 2021 Feb 1;175(2):143.}
\footnotetext[5]{Ludvigsson JF. Systematic review of COVID-19 in children shows milder cases and a better prognosis than adults. Acta Paediatrica. 2020;109(6):1088–95.}
\end{frame}

\begin{frame}{Background: Demographic and epidemiologic characteristics}
\begin{itemize}
    \item {\bf Latent period}: time between exposure and onset of infectiousness ($\approx 3$ days~\textsuperscript{1,2})
    \item {\bf Infectious period}: duration of infectiousness ($\approx 4$ days; half-maximum~\textsuperscript{1,2})
    \item {\bf Incubation period}: time between exposure and symptoms ($\approx 5$ days~\textsuperscript{1,3})
    \item {\bf Basic reproduction number}: expected number of secondary cases (2--4~\textsuperscript{1})
\end{itemize}

\footnotetext[1]{Li R et al. Substantial undocumented infection facilitates the rapid dissemination of novel coronavirus (SARS-CoV-2). Science. 2020 May 1;368(6490):489–93.}
\footnotetext[2]{He X et al. Temporal dynamics in viral shedding and transmissibility of COVID-19. Nature Medicine. 2020 May;26(5):672–5.}
\footnotetext[3]{Lauer SA et al. The Incubation Period of Coronavirus Disease 2019 (COVID-19) From Publicly Reported Confirmed Cases: Estimation and Application. Annals of Internal Medicine. 2020 Mar 10;172(9):577–82.}
\footnotetext[4]{Bar-On YM et al. SARS-CoV-2 (COVID-19) by the numbers. Eisen MB, editor. eLife. 2020 Mar 31;9:e57309}
\end{frame}

\begin{frame}{Background: distribution of children and adults across school districts}
%
\begin{figure}
    \centering
    \includegraphics[width=\textwidth]{figures/S1_fig.png}
\end{figure}
\end{frame}

\begin{frame}{Background: What is transmission?}
%
Transmission is understood in the following sense:
\begin{align*}
    \text{transmission}
    &= F(\text{susceptibility}, \text{infectivity}, \text{contact}) \\
    &= \text{probability of infection per contact} \times \text{average contact per unit time} 
\end{align*}
%
Subject to changes in
\begin{itemize}
    \item contact patterns
    \item human behavior and interventions
    \item virulence
\end{itemize}
\end{frame}

\begin{frame}{Background: Contact matrices}
\begin{figure}
    \centering
    \includegraphics[width=\textwidth]{figures/zhang2020.png}
\end{figure}

Zhang J et al. Changes in contact patterns shape the dynamics of the COVID-19 outbreak in China. Science. 2020 Jun 26;368(6498):1481–6. 
\end{frame}

%%%%% Modelling Assumptions %%%%%
\section{Modelling Assumptions}

\begin{frame}{Modelling Assumptions}
\begin{itemize}
    \item Simulated population represents a community surrounding a school.
    \item Consider two age classes: K-12 children and adults.
    \item Population further stratified into cohorts.
    \begin{itemize}
        \item Multiple rotating cohorts that trade off every week, or
        \item a hybrid approach with one in-person cohort and a remote learning cohort.
    \end{itemize}
    \item Both age classes exihibit homogeneous disease behavior in terms of {\it latency} and {\it duration of infectiousness}.
    \item Transmission between children may be higher while attending school.
    \item No vital dynamics -- no births or deaths.
    \item Individuals removed from infectious pool do not become susceptible again.
\end{itemize}
\end{frame}

\begin{frame}{Modelling Assumptions: Transmission}
%
Since we do not understand how existing evidence translates to all transmission rates, want to separate scale from type of interaction.
\begin{itemize}
    \item $\beta_{0}$: scale; aggregate property of population.
    \item $f_{ij}$: weight; strength of contribution due to $i \to j$ transmission.
\end{itemize}
Define rate of transmission from individual in age class $i$ to age class $j$ as
\begin{equation*}
    \beta_{ij} = \beta_{0} \times f_{ij}
\end{equation*}
For children, transmission modeled using a time-inhomogeneous rate by introducing a multiplier that captures the effect of increased contacts
\begin{equation*}
    \beta_{11}(t) = c(t) \times \beta_{0} \times f_{11}
\end{equation*}
\end{frame}

%%%%% SEIR Equations %%%%%
\section{Stratefied SEIR model}

\begin{frame}{SEIR: Overview}
\begin{figure}
    \centering
    \includegraphics[width=0.7\textwidth]{figures/Fig1.png}
\end{figure}
\end{frame}

\begin{frame}{SEIR: Compartments}
\begin{itemize}
    \item $S(t)$: susceptible to infection
    \item $E(t)$: exposed to infection
    \item $I(t)$: infected
    \item $R(t)$: removed; not infectious and may have additional morbidities
\end{itemize}

Expressed as fraction of population; i.e. \# in compartment / population size
\end{frame}

\begin{frame}{SEIR: Force of Infection}
Force of infection on an individual in class $j$ and cohort $k$:

\begin{eqnarray*}
    \lambda_{jk}(t) &=&
    \sum_{\text{cohort}~\ell}~\sum_{\text{age class}~i}
    \left(\parbox{6em}{\centering interaction between cohorts $k,\ell$}\right)
    \times
    \left(\parbox{6em}{\centering transmission from age group $i \rightarrow j$}\right)
    \times
    \left(\parbox{6em}{\centering fraction of infecteds in age group $i$, and cohort $\ell$}\right) \\
    &=&
    \sum_{\ell} \sum_{i}
    \alpha_{k\ell}
    \times
    \beta_{ij}
    \times
    I_{i\ell}(t).
    \end{eqnarray*}
\end{frame}

\begin{frame}{SEIR: ODEs}
\begin{alignat*}{2}
    &\text{Children} &\qquad &\text{Adults} \\
    \ddt{S_{1k}} &= - \lambda_{1k} S_{1k}                   &\qquad
    \ddt{S_{2k}} &= - \lambda_{2k} S_{2k}                   \\
    \ddt{E_{1k}} &= \lambda_{1k} S_{1k} - \sigma_{1} E_{1k} &\qquad
    \ddt{E_{2k}} &= \lambda_{2k} S_{2k} - \sigma_{2} E_{2k} \\
    \ddt{I_{1k}} &= \sigma_{1} E_{1k} - \gamma_{1} I_{1k}   &\qquad
    \ddt{I_{2k}} &= \sigma_{2} I_{2k} - \gamma_{2} I_{2k}   \\
    \ddt{R_{1k}} &= \gamma_{1} I_{1k}                       &\qquad
    \ddt{R_{2k}} &= \gamma_{2} I_{2k},
\end{alignat*}
\end{frame}

%%%%% Result 1 - child-adult transmission is key %%%%%
\section{Robustness of transmission assumptions}

\begin{frame}{Under symmetric mode of transmission}
\begin{figure}
    \centering
    \includegraphics[width=\textwidth]{figures/Fig2ABC.png}
\end{figure}

Fix $\beta_{0}=1.5$ and vary $f_{ij}$.
\end{frame}

\begin{frame}{Weak child-child transmission}
\begin{figure}
    \centering
    \includegraphics[width=\textwidth]{figures/Fig2DEF.png}
\end{figure}
\end{frame}

\begin{frame}{Weak adult-adult transmission}
\begin{figure}
    \centering
    \includegraphics[width=\textwidth]{figures/Fig2GHI.png}
\end{figure}
\end{frame}

%%%%% Result 2 - cohorts are impactful strategies %%%%%
\section{Impact of mitigation strategies}

\begin{frame}{Cohorts have a larger impact compared to targeting transmission rates}
\begin{figure}
    \centering
    \includegraphics[width=0.87\textwidth]{figures/Fig6.png}
\end{figure}
\begin{itemize}
    \item Fix $\beta_{0} = 1.2$, $f_{11} = 0.1$, $f_{12} = 0.25$, $f_{21} = 0.15$, and $f_{22} = 0.5$.
    \item Reduce $\beta_{22}$ by 40\%, then vary $\beta_{11} \times (1-s)$ and $\beta_{12} \times (1-s)$.
\end{itemize}
\end{frame}

%%%%% Result 3 - testing and reclosing are important %%%%%
\section{School reopening}

\begin{frame}{Defining a stopping time}    
\[
    t_{\text{threshold}} = \inf~
    \left\{
        t \in T : \sum_{s \in \text{window}(t)} C(s) \ge 5 \%
    \right\}
\]

\begin{itemize}
    \item Each child attending school is administered a test at each time $t$ in the test time set $T$, which we take to be every school day.
    \item The quantity $C(s) = \text{sensitivity} \times I_{1k} / q$ represents detected cases due to the active cohort $k$, adjusted for the population size of children $q$.
    \item The sum is taken over a sliding window at time $t$ extending up to 14 days in the past~\footnotemark[1]. A school closes once the 14-day prevalence in school chldren reaches $5\%$.
\end{itemize}

\footnotetext[1]{Office of Governor Gavin Newsom. Governor Gavin Newsom Lays Out Pandemic Plan for Learning and Safe Schools. 2020. https://www.gov.ca.gov/2020/07/17/governor-gavin-newsom-lays-out-pandemic-plan-for-learning-and-safe-schools/}
\end{frame}

\begin{frame}{Stopping time as a function of cohorts and initial infected}
\begin{figure}
    \centering
    \includegraphics[width=0.60\textwidth]{figures/Fig3.png}
\end{figure}
\end{frame}

\begin{frame}{Impact of testing and prevalence threshold rule}
\begin{figure}
    \centering
    \includegraphics[width=0.8\textwidth]{figures/Fig4.png}
\end{figure}
\end{frame}

\begin{frame}{Hybrid approach can avoid closure}
\begin{figure}
    \centering
    \includegraphics[width=\textwidth]{figures/Fig5.png}
\end{figure}
\end{frame}

%%%%% Conclusions %%%%%
\section{Summary}

\begin{frame}{Concluding Remarks}
Summary of results:
\begin{itemize}
    \item Child-adult transmission needs to be better understood.
    \item Minimizing contact through cohorts may be more impactful compared to direct transmission mitigation.
    \item Local prevalence is important in deciding when to reopen; the potential for increased child-child contact less so.
    \item Test sensitivity is secondary to a rapid surveillance program.
\end{itemize}

Takeaways:
\begin{itemize}
    \item A simple model can be a good platform for thinking about a problem.
    \item (Applied) mathematicians can have an impact outside of mathematics.
\end{itemize}
\end{frame}

%%%%% Acknowledgements %%%%%
\begin{frame}{Acknowledgements}
\begin{itemize}
    \item Xiang Ji, Tulane University
    \item Timothy Stutz, UCLA
    \item Jason Xu, Duke University
    \item Kenneth Lange, UCLA
    \item Mary Sehl, UCLA
    \item Janet Sinsheimer, UCLA
\end{itemize}
\end{frame}

\end{document}